%!TEX TS-program = xelatex
%!TEX encoding = UTF-8 Unicode
% Awesome CV LaTeX Template for CV/Resume
%
% This template has been downloaded from:
% https://github.com/posquit0/Awesome-CV
%
% Author:
% Claud D. Park <posquit0.bj@gmail.com>
% http://www.posquit0.com
%
%
% Adapted to be an Rmarkdown template by Mitchell O'Hara-Wild
% 23 November 2018
%
% Template license:
% CC BY-SA 4.0 (https://creativecommons.org/licenses/by-sa/4.0/)
%
%-------------------------------------------------------------------------------
% CONFIGURATIONS
%-------------------------------------------------------------------------------
% A4 paper size by default, use 'letterpaper' for US letter
\documentclass[11pt, a4paper]{awesome-cv}

% Configure page margins with geometry
\geometry{left=1.4cm, top=.8cm, right=1.4cm, bottom=1.8cm, footskip=.5cm}

% Specify the location of the included fonts
\fontdir[fonts/]

% Color for highlights
% Awesome Colors: awesome-emerald, awesome-skyblue, awesome-red, awesome-pink, awesome-orange
%                 awesome-nephritis, awesome-concrete, awesome-darknight

\definecolor{awesome}{HTML}{414141}

% Colors for text
% Uncomment if you would like to specify your own color
% \definecolor{darktext}{HTML}{414141}
% \definecolor{text}{HTML}{333333}
% \definecolor{graytext}{HTML}{5D5D5D}
% \definecolor{lighttext}{HTML}{999999}

% Set false if you don't want to highlight section with awesome color
\setbool{acvSectionColorHighlight}{true}

% If you would like to change the social information separator from a pipe (|) to something else
\renewcommand{\acvHeaderSocialSep}{\quad\textbar\quad}

\def\endfirstpage{\newpage}

%-------------------------------------------------------------------------------
%	PERSONAL INFORMATION
%	Comment any of the lines below if they are not required
%-------------------------------------------------------------------------------
% Available options: circle|rectangle,edge/noedge,left/right

\name{JooYoung}{Seo}

\position{Ph.D.~Candidate (ABD); RStudio Trusted Data Science
Instructor}
\address{Learning, Design, and Technology, 301 Keller Building,
University Park, PA 16802, USA}

\mobile{+1 814-777-5825}
\email{\href{mailto:jooyoung@psu.edu}{\nolinkurl{jooyoung@psu.edu}}}
\homepage{jooyoungseo.com}
\github{jooyoungseo}
\linkedin{jooyoungseo}
\twitter{seo\_jooyoung}

% \gitlab{gitlab-id}
% \stackoverflow{SO-id}{SO-name}
% \skype{skype-id}
% \reddit{reddit-id}

\quote{I am a learning scientist, data-science/software engineer, and
internationally certified accessibility professional.}

\usepackage{booktabs}

\providecommand{\tightlist}{%
	\setlength{\itemsep}{0pt}\setlength{\parskip}{0pt}}

%------------------------------------------------------------------------------


\usepackage{fancyhdr}
\pagestyle{fancy}
\fancyhf{}
\fancyhead[R]{\thepage}

% Pandoc CSL macros
\newlength{\cslhangindent}
\setlength{\cslhangindent}{1.5em}
\newlength{\csllabelwidth}
\setlength{\csllabelwidth}{3em}
\newenvironment{CSLReferences}[3] % #1 hanging-ident, #2 entry spacing
 {% don't indent paragraphs
  \setlength{\parindent}{0pt}
  % turn on hanging indent if param 1 is 1
  \ifodd #1 \everypar{\setlength{\hangindent}{\cslhangindent}}\ignorespaces\fi
  % set entry spacing
  \ifnum #2 > 0
  \setlength{\parskip}{#2\baselineskip}
  \fi
 }%
 {}
\usepackage{calc}
\newcommand{\CSLBlock}[1]{#1\hfill\break}
\newcommand{\CSLLeftMargin}[1]{\parbox[t]{\csllabelwidth}{#1}}
\newcommand{\CSLRightInline}[1]{\parbox[t]{\linewidth - \csllabelwidth}{#1}}
\newcommand{\CSLIndent}[1]{\hspace{\cslhangindent}#1}

\begin{document}

% Print the header with above personal informations
% Give optional argument to change alignment(C: center, L: left, R: right)
\makecvheader

% Print the footer with 3 arguments(<left>, <center>, <right>)
% Leave any of these blank if they are not needed
% 2019-02-14 Chris Umphlett - add flexibility to the document name in footer, rather than have it be static Curriculum Vitae
\makecvfooter
  {March 22, 2021}
    {JooYoung Seo~~~·~~~Curriculum Vitae}
  {\thepage}


%-------------------------------------------------------------------------------
%	CV/RESUME CONTENT
%	Each section is imported separately, open each file in turn to modify content
%------------------------------------------------------------------------------



\hypertarget{work-experience}{%
\section{Work Experience}\label{work-experience}}

\begin{cventries}
    \cventry{Software Engineer Intern}{Rstudio PBC}{Boston, MA}{May. 2020 - Aug. 2020}{\begin{cvitems}
\item Worked on accessibility improvement projects for Rstudio Server and Desktop IDE, Shiny and Rmarkdown.
\item Patched Shiny's bootstrap dependencies to improve navigation of Shiny apps for screen-reader and keyboard users (alert, tooltip, popover, modal dialog, dropdown, tab Panel, collapse, and carousel elements). \href{https://github.com/rstudio/shiny/pull/2911}{(Shiny PR \#2911)}
\item Made selectInput widget accessible by patching selectize-a11y-plugin JS library. \href{https://github.com/rstudio/shiny/pull/2993}{(Shiny PR \#2993)}
\item Developed a way to pass dynamic alt attribute for reactive plot objects in Shiny UI. \href{https://github.com/rstudio/shiny/pull/3006}{(Shiny PR \#3006)}
\item Made fontawesome and glyphicon readable to assistive technologies in Shiny UI. \href{https://github.com/rstudio/shiny/pull/2917}{(Shiny PR \#2917)}
\item Developed JS code to resolve accessibility issue in highlighted code blocks of HTML output produced by Pandoc for screen reader users. \href{https://github.com/rstudio/rmarkdown/pull/1833}{(Rmarkdown PR \#1833)}
\item Authored technical documents on \href{https://support.rstudio.com/hc/en-us/articles/360049776974-Using-RStudio-Server-in-Windows-WSL2}{how to run RStudio Server via Windows Subsystem for Linux} and \href{https://support.rstudio.com/hc/en-us/articles/360045612413-RStudio-Screen-Reader-Support}{RStudio Screen Reader Support.}
\end{cvitems}}
    \cventry{Co-Founder and Project Manager}{ICE Soft}{Seoul, South Korea}{Jul. 2010 - Jun. 2011}{\begin{cvitems}
\item Co-founded and managed a start-up company to develop Android-based navigation App and assistive technology for blind people.
\item Applied for and received a \$30,000 fund from the city of Seoul.
\end{cvitems}}
\end{cventries}

\hypertarget{education}{%
\section{Education}\label{education}}

\begin{cventries}
    \cventry{Ph.D. in Learning, Design, and Technology}{The Pennsylvania State University}{University Park, PA}{Aug. 2016 - Present}{\begin{cvitems}
\item Expected graduation: June 2021.
\item Dissertation Title: ``Discovering Informal Learning Cultures of Blind Individuals Pursuing STEM Disciplines: A Quantitative Ethnography Using Public Listserv Archives.''
\item Committee members: Drs. Gabriela T. Richard (adviser; dissertation chair), Roy B. Clariana, ChanMin Kim, and Mary Beth Rosson.
\end{cvitems}}
    \cventry{M.Ed. in Learning, Design, and Technology}{The Pennsylvania State University}{University Park, PA}{Aug. 2014 - May. 2016}{\begin{cvitems}
\item GPA 3.97/4.0.
\end{cvitems}}
    \cventry{Double B.A. in Education, English Literature}{Sungkyunkwan University}{Seoul, South Korea}{Mar. 2009 - Feb. 2014}{\begin{cvitems}
\item GPA 4.08/4.50.
\end{cvitems}}
\end{cventries}

\hypertarget{skills}{%
\section{Skills}\label{skills}}

\begin{cvskills}
  \cvskill
    {Braille}
    {UEB (Unified English Braille), Nemeth (Math), Korean, Japanese Brailles}

  \cvskill
    {Data Science}
    {R (advanced: published 3 packages in CRAN), Python (advanced: tensorflow/keras/scikit-learn), SQL}

  \cvskill
    {Reproducible Report}
    {Markdown/Rmarkdown, R shiny apps, Jupyter Notebook, LaTeX, Pandoc, lua}

  \cvskill
    {DevOps}
    {Git, Docker, AWS, Travis CI, Cygwin}

  \cvskill
    {Front-End}
    {HTML/CSS/JS/PHP/ARIA (advanced level to design according to WCAG 2.0/2.1), WordPress, Drupal, Hugo}

  \cvskill
    {Back-End}
    {Unix/Linux Shell scripts, LAMP, Django, REST API}

  \cvskill
    {Compile Programming Languages}
    {C/C++/C\#, Java, Quorum}

  \cvskill
    {Quantitative Research}
    {t-test, within-/between-subjects/Repeated Measures ANOVAs/ANCOVAs/MANOVAs/MANCOVAs, Regressions, \newline HLM, Factor Analysis, Network Analysis, SEM, PCA, Unsupervised/Supervised Machine Learning}

  \cvskill
    {Qualitative Research}
    {Virtual Ethnography, Content Analysis, Grounded Theory, Phenomenology, Case Studies, Interaction Analysis}

  \cvskill
    {Mixed Research}
    {Text Mining, Quantitative Ethnography, Explanatory/Exploratory Sequential Mixed Methods}

  \cvskill
    {Languages}
    {English/Korean}
\end{cvskills}

\hypertarget{publications}{%
\section{Publications}\label{publications}}

\hypertarget{refereed-journal-papers}{%
\subsection{Refereed Journal Papers}\label{refereed-journal-papers}}

\hypertarget{refs_journals}{}
\leavevmode\vadjust pre{\hypertarget{ref-doi:10.1080ux2f09286586.2020.1863993}{}}%
Choi, S., \& \textbf{Seo, J.} (2020). Trends in healthcare research on
visual impairment and blindness: Use of bibliometrics and hierarchical
cluster analysis. \emph{Ophthalmic Epidemiology}, \emph{0}(0), 1--36.
\url{https://doi.org/10.1080/09286586.2020.1863993}. \emph{PMID:
33380253}.

\leavevmode\vadjust pre{\hypertarget{ref-seo2019maker}{}}%
\textbf{Seo, J.} (2019). Is the maker movement inclusive of {ANYONE}?:
Three accessibility considerations to invite blind makers to the making
world. \emph{{TechTrends}}, \emph{63}(5), 514--520.
\url{https://doi.org/10.1007/s11528-019-00377-3}

\leavevmode\vadjust pre{\hypertarget{ref-seo2019arow}{}}%
\textbf{Seo, J.}, \& McCurry, S. (2019). LaTeX is NOT easy: Creating
accessible scientific documents with r markdown. \emph{Journal on
Technology and Persons with Disabilities}, \emph{7}, 157--171.

\hypertarget{papers-in-refereed-conference-proceedings}{%
\subsection{Papers in Refereed Conference
Proceedings}\label{papers-in-refereed-conference-proceedings}}

\hypertarget{refs_proceedings}{}
\leavevmode\vadjust pre{\hypertarget{ref-seo2020coding}{}}%
\textbf{Seo, J.}, \& Richard, G. T. (2020). Coding through touch:
Exploring and re-designing tactile making activities with learners with
visual dis/abilities. In M. Gresalfi \& I. Horn (Eds.),
\emph{Interdisciplinarity in the learning sciences, 14th international
conference of the learning sciences (ICLS) 2020} (Vol. 3, pp.
1373--1380). Nashville, TN: International Society of the Learning
Sciences (ISLS).

\leavevmode\vadjust pre{\hypertarget{ref-seo2019discovering}{}}%
\textbf{Seo, J.} (2019). Discovering informal learning cultures of blind
individuals pursuing STEM disciplines: A quantitative ethnography using
listserv archives. \emph{The first international conference on
quantitative ethnography: Doctoral consortium}, S66--S67. Madison, WI.
\emph{Awarded the best Doctoral Consortium Proposal Cengage fellowship}.

\leavevmode\vadjust pre{\hypertarget{ref-seo2018making}{}}%
\textbf{Seo, J.} (2018). Accessibility and inclusivity in making:
Engaging learners with all abilities in making activities. In
\emph{Proceedings of the 3rd learning sciences graduate student
conference} (pp. 141--142). Nashville, TN: LSGSC Planning Team.

\leavevmode\vadjust pre{\hypertarget{ref-seo2018accessibility}{}}%
\textbf{Seo, J.}, \& Richard, G. T. (2018). Accessibility, making and
tactile robotics: Facilitating collaborative learning and computational
thinking for learners with visual impairments. In J. Kay \& R. Luckin
(Eds.), \emph{Rethinking learning in the digital age: Making the
learning sciences count, 13th international conference of the learning
sciences (ICLS) 2018} (Vol. 3, pp. 1755--1757). London, UK:
International Society of the Learning Sciences (ISLS).

\leavevmode\vadjust pre{\hypertarget{ref-konecki2017role}{}}%
Konecki, M., Lovrenčić, S., \textbf{Seo, J.}, \& LaPierre, C. (2017).
The role of ICT in aiding visually impaired students and professionals.
\emph{Proceedings of the 11th Multidisciplinary Academic Conference},
148.

\leavevmode\vadjust pre{\hypertarget{ref-seo2017embracing}{}}%
\textbf{Seo, J.}, AlQahtani, M., Ouyang, X., \& Borge, M. (2017).
Embracing learners with visual impairments in CSCL. In B. K. Smith, M.
Borge, E. Mercier, \& K. Y. Lim (Eds.), \emph{Making a difference:
Prioritizing equity and access in CSCL, 12th international conference on
computer supported collaborative learning (CSCL) 2017} (Vol. 2, pp.
573--576). Philadelphia, PA: International Society of the Learning
Sciences (ISLS).

\hypertarget{data-science-package-developments}{%
\subsection{Data Science Package
Developments}\label{data-science-package-developments}}

\hypertarget{refs_R_packages}{}
\leavevmode\vadjust pre{\hypertarget{ref-R-youtubecaption}{}}%
\textbf{Seo, J.}, \& Choi, S. (2020). \emph{Youtubecaption: Downloading
YouTube subtitle transcription in a tidy tibble data frame}. Retrieved
from \url{https://CRAN.R-project.org/package=youtubecaption}. \emph{Over
9543 download}.

\leavevmode\vadjust pre{\hypertarget{ref-R-ezpickr}{}}%
\textbf{Seo, J.}, \& Choi, S. (2019). \emph{Ezpickr: Easy data import
using GUI file picker and seamless communication between an excel and
r}. Retrieved from \url{https://CRAN.R-project.org/package=ezpickr}.
\emph{Over 19K download}.

\leavevmode\vadjust pre{\hypertarget{ref-R-mboxr}{}}%
\textbf{Seo, J.}, \& Choi, S. (2019). \emph{Mboxr: Reading, extracting,
and converting an mbox file into a tibble}. Retrieved from
\url{https://CRAN.R-project.org/package=mboxr}. \emph{Over 15K
download}.

\end{document}
